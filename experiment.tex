\section{Experiment}
While the modified Count-min sketch algorithm provides an error bound for the
estimated movie averages, what we are really interested in is how much it
affects the ordering of the movies when sorted by these estimated averages
compared to the real averages. In this section, we evaluate the modified
Count-min sketch algorithm by running it on real-life data, and comparing the
order it produces with the correct order. 

We quantify the error in ordering by the Kendall tau distance. For a permutation
of movies $\pi$, we say that $x <_\pi y$ if $x$ comes before $y$ in $\pi$. The
Kendall tau distance between two orderings of our movie set is then $\mathrm{KT}
\left(\pi_1,\pi_2\right) = \left|\{x,y\}: x <_{\pi_1} y \wedge y <_{\pi_2}
x\right|$. The Kendall distance is often normalized to the maximum number of
inversions between the orderings, $n(n-1)/2$ for $n$ elements. We expect random
orderings to have a normalized Kendall tau distance of $\frac{1}{2}$, so
only results with a lower distance can be considered useful.

Our data is sampled from the \texttt{Netflix Prize} data set.
The full data set contains more than 100 million user ratings, for more than 17
thousand movies.
We test the algorithm on two samples of the \texttt{Netflix} data, with 100
million and 50 million ratings respectively. The ratings are chosen from the
dataset as follows: Go through the set of movies in order of decreasing rating
count. Pick each movie, until the desired number of ratings is encountered.

\pgfplotsset{scaled x ticks=false}
\begin{center}
\begin{tikzpicture}
\begin{axis}[
	title=Count-min order error,
	xlabel={Error bound $\varepsilon$},
	ylabel={KT. norm.},
	legend pos=south east,
	xticklabel style={
		/pgf/number format/.cd,
		fixed,
		fixed zerofill,
		precision=2,
		/tikz/.cd
	},
]
\addplot table [y=100M,x=E]{allresults};
\addlegendentry{100M ratings}
\addplot table [y=50M,x=E]{allresults};
\addlegendentry{50M ratings}
\end{axis}
\end{tikzpicture}
\end{center}
