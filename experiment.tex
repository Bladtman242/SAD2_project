\section{Experiment}
While the modified Count-min sketch algorithm provides an error bound for the
estimated movie averages, what we are really interested in is how much it
affects the ordering of the movies when sorted by these estimated averages
compared to the real averages. In this section, we evaluate the modified
Count-min sketch algorithm by running it on real-life data, and comparing the
order it produces with the correct order. 

We quantify the error in ordering by the Kendall tau distance. For a permutation
of movies $\pi$, we say that $x <_\pi y$ if $x$ comes before $y$ in $\pi$. The
Kendall tau distance between two orderings of our movie set is then $\mathrm{KT}
\left(\pi_1,\pi_2\right) = \left|\{x,y\}: x <_{\pi_1} y \wedge y <_{\pi_2}
x\right|$. The Kendall distance is often normalized to the maximum number of
inversions between the orderings, $n(n-1)/2$ for $n$ elements. We expect random
orderings to have a normalized Kendall tau distance of $\frac{1}{2}$, so
only results with a lower distance can be considered useful.

\subsection{Results}
Our data is sampled from the \textit{Netflix
Prize}(http://academictorrents.com/details/9b13183dc4d60676b773c9e2cd6de5e5542cee9a) data set.
The full data set contains more than 100 million user ratings, for more than 17
thousand movies.
We test the algorithm on three samples of the \texttt{Netflix} data:
\begin{itemize}
	\item \textit{100m ratings} contains all ratings for the $13726$ most
		frequently rated movies. Roughly 100 million ratings total,
	\item \textit{50Mmin ratings} contains all ratings for the $17146$ least
		frequently rated movies. Roughly 50 million ratings total,
	\item \textit{50m ratings} contains all ratings for the $611$ most
		frequently rated movies. Roughly 50 million ratings total,
	\item \textit{min10K ratings} contains all ratings for the $2042$ movies that have more
		than $10.000$ ratings.

\end{itemize}

\pgfplotsset{scaled x ticks=false}
\begin{center}
\begin{tikzpicture}
\begin{axis}[
	title=Count-min order error,
	xlabel={Error bound $\varepsilon$},
	ylabel={KT. norm.},
	legend pos=south east,
	xticklabel style={
		/pgf/number format/.cd,
		fixed,
		fixed zerofill,
		precision=2,
		/tikz/.cd
	},
]
\addplot table [y=100M,x=E]{allresults};
\addlegendentry{\textit{100M ratings}}
\addplot table [y=50M,x=E]{allresults};
\addlegendentry{\textit{50M ratings}}
\addplot table [y=min10K,x=E]{allresults};
\addlegendentry{\textit{min10K ratings}}
\addplot table [y=50Mmin,x=E]{allresults};
\addlegendentry{\textit{50Mmin ratings}}
\end{axis}
\end{tikzpicture}
\end{center}

\subsection{100m ratings}		
As we can see the first plot - $100M raitings$ show how badly algorithm worked for this data set.
The Kendall-Tau value starts from $0.35$ and rapidly grows in direction to $0.5$. Which is very bad.
This behaviour is caused by length of the stream - longer it is more worse results we get. And by the fact
that the 100 millions is basicaly our entire data set. So the number of raitings per movie differs.

\subsection{50Mmin ratings}	
Same trends as in $100m raitings$ is present in $50Mmin raitings$. Again we are working on relatively big set. But this time it contains
only the movies which were least rated. Even with the fact we are working on half smaller set we still gets poor results. We cannot 
achieve good result with set which contains rarely rated movies.

\subsection{50m ratings}
Again we are working on smaller set. But this time we have the most frequently rated movies. Difference in shape of plot is huge.
It starts from Kendall-Tau distance equal $0$ where the previous two data sets started with 0.3 more. Even for higher error bounds 
the results ale still better. 

\subsection{min10K ratings}
Last which produced best results. Which contains only $10$ tousands movies with high amount of raitings. Shape of plot is similar to 
$50min raitings$ but slight better by average $0.20$ normalized Kendall-Tau distance. Probably thanks to fact the stream is smaller than
$50m raitings$. 



\subsection{Implementation}
The implementation can be found in \texttt{code} folder that came with this
report. We Note that the implementation does not live up to the performance
bounds stated in section \ref{sec:sketching}. This is not a problem, since we
are assessing the correctness of the algorithm, not it's throughput. The
implementation differs in two ways: To emulate querying all data points, a list
of all observed movies is kept, requiring $O(|U|)$ extra memory. Furthermore the
ratings are sorted when queried, rather than maintaining the ordering
dynamically. Again, this does not effect the experiment.
