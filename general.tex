\section*{General approach}
One of the basic and also efficient concepts is using a data structure which
will provide mechanism for extracting interesting data efficiently. 

\noindent \textbf{Binary Search Tree}

Creation of BST is an insertion of each element in $O(\log{}n)$ time. As a result 
we get sorted structure of our inputs - accessing all of them takes $O(n)$.
Using this approach requires to store all input elements - which consumes $O(n)$ space. 
And if we want to find Low k elements of set we still need to travers all set.

\noindent \textbf{Heap and Min-Max Heap}

Ordering items using Heap takes $O(n\log{}n)$. Where creation of a heap takes
$O(n)$. Then we are traversing through heap n times using fuction which takes
$O(\log{}n)$. Which results in $O(n + n\log{}n)$, which leads to our prevoiusly
stated complexity.

In scenario when we are only interested in Top and Low k elements of a set we might takes
different heap based approach called Min-Max Heap.

Provides properties of regular heap like a: 
\begin{itemize}
\item $O(n)$ built time,
\item $O(\log{}n)$ for insertion and deletion
\end{itemize}

But even more important are properties carried by Min-Max variation. Which are 
guaranteeing us a constant time for Min and Max operations. The Top/Low(k) 
finding operation can be performed in $O(k\log{}k)$ time.

Using this approach requires to store all input elements. It takes $O(n)$ for 
large sets.

Based on : \texttt{M. D. Atkinson, J.R. Sack, N. Santoro, and T. Strothotte, 
Communications of the ACM, 
October, 1986, 
Min-Max Heaps and Generalized Priority Queues}
