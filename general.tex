\section*{General approach}
To tacle this problem we can utilize two kinds of approaches. 
One is sorting entire set. Main adventage is the simplicity. 
Sorting is one of the basic, fundamentals algorithmic problems.
But also carries a drawbacks. Main one is requirement to sort 
each time when data set in altered. 
Also sorting algorithm effciency is often bounded to the current 
state of input. 


Second concept is using a data structure which are able to produce
sorted input once they are built. The main part in a construction
- every time structure is changed we need to execute this operation
with maintaining or restoring it's properties.  
 
\noindent \textbf{Binary Search Tree}

From \texttt{Algortihms, 4 Edition}:
\begin{quote}
"A binary search tree (BST) is a binary tree where each node has a Comparable key 
(and an associated value) and satisfies the restriction that the key in any node is larger than the keys 
in all nodes in that node's left subtree and smaller than the keys in all nodes in that node's right subtree."
\end{quote}

Creation of BST is an insertion of each element in $O(\log{}n)$ time. As a result 
we get sorted structure of our inputs - accessing all of them takes $O(n)$.
Using this approach requires to store all input elements - which consumes $O(n)$ space. 
And if we want to find Low k elements of set we still need to travers all set.

\noindent \textbf{Heap and Min-Max Heap}

From \texttt{Heap, Wikipedia}:
\begin{quote}Heap is a specialized tree-based data structure that satisfies the heap property: 
If A is a parent node of B then the key of node A is ordered with respect to the key of node 
B with the same ordering applying across the heap.
\end{quote}

Ordering items using Heap takes $O(n\log{}n)$. Where creation of a heap takes
$O(n)$. Then we are traversing through heap n times using fuction which takes
$O(\log{}n)$. Which results in $O(n + n\log{}n)$, which leads to our prevoiusly
stated complexity.

Also Binary Heap can offer better average insertion time than BST which may
be important factor in some scenarions when the data are often updated but we 
still need to able to acess the sorting results.

In scenario when we are only interested in Top and Low k elements of a set we might takes
different heap based approach called Min-Max Heap.

Provides properties of regular heap like a: 
\begin{itemize}
\item $O(n)$ built time,
\item $O(\log{}n)$ for insertion and deletion
\end{itemize}

From \texttt{Min-max heap, Wikipedia}:
\begin{quote}Heap is a specialized tree-based data structure that satisfies the heap property: 
If A is a parent node of B then the key of node A is ordered with respect to the key of node 
B with the same ordering applying across the heap.
\end{quote}

But even more important are properties carried by Min-Max variation. Which are 
guaranteeing us a constant time for Min and Max operations. The Top/Low(k) 
finding operation can be performed in $O(k\log{}k)$ time.

Using this approach requires to store all input elements. It takes $O(n)$ for 
large sets.

Based on : \texttt{M. D. Atkinson, J.R. Sack, N. Santoro, and T. Strothotte, 
Communications of the ACM, 
October, 1986, 
Min-Max Heaps and Generalized Priority Queues}
