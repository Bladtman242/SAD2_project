\section*{General approach}
One of the basic and also efficient concepts is using a data structure which
will provide mechanism for extracting interesting data efficiently. 
Natural choice and one of the simplest is a heap. But our specific problem of 
identifing Top/Low elements of a set requires a special capabilities.

\noindent \textbf{Min-Max Heap}

Provides properties of regular heap like a: 
\begin{itemize}
\item $O(n)$ built time,
\item $O(log{}n)$ for insertion and deletion
\end{itemize}

But even more important are properties carried by Min-Max variation. Which are 
guaranteeing us a constant time for Min and Max operations. The Top/Low(k) 
finding operation can be performed in $O(k\log{}k)$ time.

Using this approach requires to store all input elements. It takes $O(n)$ for 
large sets.

Based on : \texttt{M. D. Atkinson, J.R. Sack, N. Santoro, and T. Strothotte, 
Communications of the ACM, 
October, 1986, 
Min-Max Heaps and Generalized Priority Queues}