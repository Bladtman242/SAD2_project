\section*{General approach extension}
\noindent \textbf{Parallelization}

Utilizing power of multicore architecture in this scope is very tempting.
But our current approach is not supportive. By using data structure like Binary Tree
or Heap in a single thread scenario we are receiving multiple benefits like logarithmic time 
for insertion for each element or limited space usage. For each new element we have to only perform
insertion with logarithmic cost. For scenario when we are interested in intermidiate results it is 
a very big adventage against sorting algorithms like QuickSort or MergeSort.

But these algorithms have adventage over data structure based sorting - they are much
more prone to Parallelization. Depending on the architecture or use case even in light 
of need processing partial input it might be profitiable to resort our data multiple
time using benefits of multicore, multithreaded computing if time is our main constraint.

Creation of BST is an insertion of each element in $O(\log{}n)$ time. As a result 
we get sorted structure of our inputs - accessing all of them takes $O(n)$.
Using this approach requires to store all input elements - which consumes $O(n)$ space. 
And if we want to find Low k elements of set we still need to travers all set.

