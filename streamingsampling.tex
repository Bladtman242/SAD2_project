\section*{A Stream Based Approach}
Streaming algorithms provide excellent solutions to many problems where data
sets are large enough that we wish (or need) to sacrifice correctness for low
memory usage and time consumption.
From \texttt{Ikonomovska\--Zelke}:
\begin{quote}
Streaming algorithms drop the demand of random access to the input. Rather, the input
is assumed to arrive in arbitrary order as an input stream. Moreover, streaming algorithms
are designed to settle for a working memory that is much smaller than the size of the input.
\end{quote}
To be precise, they require that the size of the working memory is sublinear in both the cardinality of the stream and the universe.
Due to this nature of streaming algorithms they are not commonly used for
problems that require analysing parts of non-constant, non-parameterized size
of the data set, for each given input. Hence an approach to
solve our problem --- sorting -- based on streaming algorithms will provide some interesting
trade-offs.

We begin with a definition of our stream, and a simple algorithm for our
problem. Our input is a turnstile stream. The universe
$U$ is the set of movies in our database, and each stream item is a pair
$(j,r) \in U\times \left(\left[1,10\right]\cap \mathbb{N}\right)$.
With our definition of movie ranks the turnstile stream is strict, and furthermore
the rank $r$ for every element is positive. This is not a necessary restriction
however.

The simplest algorithm to solve our problem is then simply calculating the
frequency vector for the stream, and sort it when our algorithm is queried.
However, this algorithm is not very satisfactory. The working memory is
sublinear in the cardinality of the stream, but not in the universe size. It
does not provide a current solution either, as we have to sort the frequency
vector when queried. On the positive side, the solution provided by the
algorithm is correct. To be precise, this algorithm would require $O(m)$ working
memory, and time for each stream item, $m$ being the number of distinct movies
in the stream, or the universe size $|U|$. A query would then require
$O(n \log(m))$ time, which is unacceptable.

\begin{enumerate}
	\item{Maintaining ordering with each item (probably based on b-tree or
		binary-heap or similar idea)}
	\item{Approximation based on sampling (at least two approaches: sample
		movies (dangerous) or sample ratings (less dangerous, but
	provides no adv. wrt working memory))}
	\item{Maybe sketching/hashing approach to approximation?}
\end{enumerate}

For 1 --- maintaining ordering --- 
