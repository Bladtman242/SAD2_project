\section{Map Reduce Approach}
In light of constantly growing datasets solving even 
most basic computational problems becomes more difficult and complex.
One of the attempts to adress this need is a model called MapReduce.
Which allows processing in a parallel, distributed manner.

\section{MapReduce}
MapReduce takes advantage of locality of data - by processing it on 
or near the storage assets in order to minimize network, space, IO,
CPU cost on each machine.
Execution in conducted in rounds. Each one of them three phases:

\begin{itemize}
\item Map: preprocess the data which will be delievered to each machine by applying 
given function to each element. 
\item Shuffle: it is responsible for actual data transfer. Data redistribiution is based on 
keys generated in Map phase. Data belonging to one key are directed to same machine. 
\item Reduce: each machine process given load. There is no network traffic or sharing between machines
in Reduce phase.
\end{itemize}


\section{Two subproblems}
Our problem of creating ranking of movies based on the user raiting 
consist two subproblems. First one is creation of averages movie rate.
Second is a putting them in order. 


\subsection{Sorting average raitings}
In 2008 Owen O'Malley won TeraByte sorting competition using Apache Hadoop
- Java framework which process data using MapReduce. His solution - called TeraSort offers
high scalability without adding overhead and overall complexity.
In paper Minimal MapReduce Algorithms TeraSort was classified as
minimal MapReduce algorithm. Which definition and properties are given
here:

From \texttt{Minimal MapReduce Algorithms}:
\begin{quote}
Denote by $S$ the set of input objects 
for the underlying problem. Let n, the problem cardinality,
be the number of objects in $S$, and $t$ be the number of machines
used in the system. Define $m = n/t$, namely, $m$ is the number
of objects per machine when $S$ is evenly distributed across the
machines. Consider an algorithm for solving a problem on $S$.
We say that the algorithm is minimal if it has all of the following
properties.

\begin{itemize}
\item Minimum footprint: at all times, each machine uses only
O(m) space of storage.
\item \emph{Bounded net-traffic}: in each round, every machine sends
and receives at most $O(m)$ words of information over the
network.
\item  \emph{Constant round}: the algorithm must terminate after a
constant number of rounds.
\item \emph{optimal computation}: every machine performs only
$O(T_seq /t)$ amount of computation in total (i.e., summing
over all rounds), where $T_seq$ is the time needed to solve the
same problem on a single sequential machine. Namely, the
algorithm should achieve a speedup of t by using t machines
in parallel. 
\end{itemize}
\end{quote}


Which perfectly denotes main advantages of the TeraSort. The minimum footprint property guarantees
that each of our machines will use the same amount of memory. Which makes profiling
and planning infractures much easier. The same hold for network throughput - by knowing
upper bound how much data will be transferred in the process we easily can track bottlenecks
and scale our architecture.
By having our load balanced it also benefits to performance. We can easily adopt amount of machines
taking part in the computation which allows to easily maintain balance between speed and costs.
Of course we could just use Quicksort instead of TeraSort - but the constant amount of rounds property
is highly in favour of TeraSort. Shuffling is very expensive operation - by limiting amount to constant
factor we are effectively it is impact or overall performance.


\subsection{Calculating average ranking}
To calculate Moving Average of each movie we again will used TeraSort.
But with modified Round 2. Where instead of sorting items in Reduce phase
we will calculate average. Since TeraSort by design is prepared to execute
sorting procedure in this particular phase transition is very simple.

Thanks to fact that each machine contains part of data. And this particular part contains movies from
a given range it simplifies process a lot. And makes it faster because we do not need to iterate over entire
input just in specific part.
Our Moving Average calculation procedure is very simple. Similar to General Approach
we are using HashMap. Where every new encountered movie is added to HashMap. If it is already
present we are adding next rating to property which holds sum of it and also incrementing counter.
Which are used to calculation average.







