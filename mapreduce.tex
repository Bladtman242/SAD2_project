\chapter{Extension}
\section{A Distributed Sort}
Many problems can be solved faster by parallelizing computation. At the extreme
end of the parallelization spectrum lies distributed computing, where
computation is distributed not among cores of a processor, but among individual
computers, each with their permanent storage, memory and processors.

MapReduce is model for distributing computation across many (hundredths or
thousands) of computers. This distribution often comes with a significant
overhead, and so it usually only pays off for data sets that are too large to
store on a single machine. It seems improbable that such a data set should exist
for movie ratings. The full \texttt{IMDB} data set consists of less than
100 billion ratings ($\sim 1.5 \times 10^6$, the most frequently rated of which
have $\sim 1.5 \times 10^6$ ratings each). A rating is a number
$\left(\left[1,10\right]\cap \mathbb{N}\right)$, which can be stored in
$\lceil \log(10)\rceil = 4$ bits, which means the full set of ratings can be
stored in less than $47$ GB. However, our \texttt{IMDB} motivation can be
directly translated to e.g. sensory network data, with the possible exception of
the small domain size, which we will not rely on in this section.

We describe a simple MapReduce algorithm that provides an exact solution to our
sorted-averages problem, and does so in a constant number of rounds. To achieve
constant-round sorting, we use the TeraSort algorithm \texttt{Owen O'Mally
(2008)}\citep{terabytesort} \texttt{Yufei Tao et
al.(2013)}\citep{minimalmapreduce}.

\subsection{MapReduce}
MapReduce takes advantage of locality of data - by processing it on 
or near the storage assets in order to minimize network, space, IO, CPU cost on each machine.
A single MapReduce round consists of three phases:

\begin{description}
	\item [Mapping] A function, or \textit{mapping}, is executed on each $(key,
		value)$ pair, resulting in a new set of $(key,value)$ pairs
	\item [Shuffling] The pairs emitted in the mapping phase are redistributed among
		the machines. Pairs are distributed so that every pair the
		same key is send to the same machine. We will rely on a stronger
		guarantee.
	\item [Reducing] A reducer, or \textit{fold}, is applied to each $(key,
		\{values\})$ pair.
\end{description}

Generally, the map and reduce functions are provided by the programmer, and the
framework handles shuffling. The TeraSort algorithm relying on the shuffling
phase to distribute keys so that the machines constitute a partial sort over the
keys. That is, for a sorted sample $samp$ over the keys, where $|samp|$ is equal to the
number of reducers, all keys such that $samp_{i-1} \le key \le samp_i$ are send
to the $i$th reducer \citep{minimalmapreduce} 

\subsection{Two subproblems}
Our problem of creating ranking of movies based on the user raiting 
consist two subproblems. First one is creation of averages movie rate.
Second is a putting them in order. 

\subsection{Calculating average ranking}
To calculate Moving Average of each movie we again will used TeraSort.
But with modified Round 2. Where instead of sorting items in Reduce phase
we will calculate average. Since TeraSort by design is prepared to execute
sorting procedure in this particular phase transition is very simple.

\subsection{Sorting using TeraSort}
Thanks to fact that each machine contains part of data. And this particular part contains movies from
a given range it simplifies process a lot. And makes it faster because we do not need to iterate over entire
input just in specific part.
Our Moving Average calculation procedure is very simple. Similar to General Approach
we are using HashMap. Where every new encountered movie is added to HashMap. If it is already
present we are adding next rating to property which holds sum of it and also incrementing counter.
Which are used to calculation average.

\subsection{Sorting average raitings}
In 2008 Owen O'Malley won TeraByte sorting competition using Apache Hadoop
- Java framework which process data using MapReduce. His solution - called TeraSort offers
high scalability without adding overhead and overall complexity.

First the algorithm starts with dividing load between each machine. Then takes sample 
out of each one and sends to one given machine where they are sorted.
Then utilize it in order to create "boundary objects" and redistribiutes them
to all machines. Using the infromation contained in boundary objects 
machine are exchaning data from local storages.
One that is done each machine is sorting local data which completes process of sorting.

In paper Minimal MapReduce Algorithms TeraSort was classified as
minimal MapReduce algorithm. Which definition and properties are given
here:


\texttt{Minimal MapReduce Algorithms}\citep{minimalmapreduce}:

"Denote by $S$ the set of input objects 
for the underlying problem. Let n, the problem cardinality,
be the number of objects in $S$, and $t$ be the number of machines
used in the system. Define $m = n/t$, namely, $m$ is the number
of objects per machine when $S$ is evenly distributed across the
machines. Consider an algorithm for solving a problem on $S$.
We say that the algorithm is minimal if it has all of the following
properties.

\begin{itemize}
\item \emph{Minimum footprint}: at all times, each machine uses only
O(m) space of storage.
\item \emph{Bounded net-traffic}: in each round, every machine sends
and receives at most $O(m)$ words of information over the
network.
\item  \emph{Constant round}: the algorithm must terminate after a
constant number of rounds.
\item \emph{optimal computation}: every machine performs only
$O(T_seq /t)$ amount of computation in total (i.e., summing
over all rounds), where $T_seq$ is the time needed to solve the
same problem on a single sequential machine. Namely, the
algorithm should achieve a speedup of t by using t machines
in parallel." 
\end{itemize}


\subsection{Benefits of TeraSort}
Which perfectly denotes main advantages of the TeraSort. The minimum footprint property guarantees
that each of our machines will use the same amount of memory. Which makes profiling
and planning infractures much easier. The same hold for network throughput - by knowing
upper bound how much data will be transferred in the process we easily can track bottlenecks
and scale our architecture.
By having our load balanced it also benefits to performance. We can easily adopt amount of machines
taking part in the computation which allows to easily maintain balance between speed and costs.
Of course we could just use Quicksort instead of TeraSort - but the constant amount of rounds property
is highly in favour of TeraSort. Shuffling is very expensive operation - by limiting amount to constant
factor we are effectively it is impact or overall performance.
