\section*{Online sorting}
From \texttt{Algorithms and Theory of Computation Handbook}:
\begin{quote}
An algorithm that must process each input in turn, without detailed knowledge
of future inputs.
\end{quote}

Not every online algorithm has an offline counterpart. And often also online
algorithms cannot match the performance of offline algorithms.
But in our scenario we want to create competive, suitable solution - the online
algorithm seems to be great fit.

\begin{itemize}
	\item{Storing in right order each input batch on heap.}
	\item{Taking adventage of algorithm like Insertion Sort}
	\item{Partial sotring,Odds algorithm - which can be valueable for defining optimal stopping}
  \item{By combining techniques listed above. }
\end{itemize}

With n inputs we are creating a heap - which takes n log n operations. This gives
us a required data. Then if the next input arives the algorithm inserts each
of new input with log n. If the item is already on the heap then we are changing
it average. So each input is processed in input size log input size.
Which is very good score but we have to story each unique movie.

Insertion sorts allows us to eaisly distribiute results into multiple locations
which could be adventage for very large sets.

Odds algorithm, Partial sorting - approach is tempting but it is compromising
the quality of output.  
