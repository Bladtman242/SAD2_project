\documentclass[a4paper, titlepage]{report}

\usepackage[utf8]{inputenc}
\usepackage{courier} % Required for the courier font
\usepackage[bookmarks]{hyperref}

%redefine percentage sign to be a little smaller
\let\oldpct\%
\renewcommand{\%}{\scalebox{.9}{\oldpct}}
\begin{document}

\title{Problem Description}
\author{
	Radoslaw Niemczyk
	\\\texttt{radn@itu.dk}
	\and
	Sigurt Bladt Dinesen
	\\\texttt{sidi@itu.dk}
}

\maketitle

%\tableofcontents

\section*{Ranking a large data set}
Ranking has become an increasingly important problem, with ever-growing datasets
both in the industry and in academia, the ability to work with large ranked
sets, and to do so fast.

Ranking is used commercially to provide costumers with proposed subsets of
products, to make it easier to choose a product of interest.

This project is concerned with sorting a large set of user rankings.
In particular, we will explore the use of the following techniques:

\begin{description}
	\item[Parallelization] 

	\item[Randomization] 

	\item[Approximation] We would like to explore the possibility of
		performing an approximate sort. User--provided rankings are
		often inconsistent. For instance, if a user ranks a move $m$
		lower than a movie $m'$, does that mean he liked $m$ less, or
		that he likes that genre less? It is unclear weather or not the
		scale is linear, and the same user may have different
		experiences on different days. This means that an \textit{exact}
		ordering might not be necessary. If it provides a speedup, it
		may be well worth it to perform a partial sorting of the
		rankings --- such that a top--ten might really be a top--eight,
		plus 12 and 14.

	\item[Online sorting] There are several possible benefits from online
		sorting. For slow data sources it may result in a speedup,
		exploiting the efficiency of certain sorting algorithms when
		dealing with partially ordered data, to lessen the time spend
		waiting for the data. The 

\end{description}

\end{document}

