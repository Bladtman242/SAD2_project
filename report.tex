\documentclass[a4paper, titlepage]{article}
\usepackage[utf8]{inputenc}
\usepackage{pgfplots}
\usepackage{tikz}
\pgfplotsset{compat=1.5}
\usepackage{courier} % Required for the courier font
\usepackage[bookmarks]{hyperref}
\usepackage{amsfonts}

%redefine percentage sign to be a little smaller
\let\oldpct\%
\renewcommand{\%}{\scalebox{.9}{\oldpct}}
\begin{document}

\title{SAD2 exam report}
\author{
	Radoslaw Niemczyk
	\\\texttt{radn@itu.dk}
	\and
	Sigurt Bladt Dinesen
	\\\texttt{sidi@itu.dk}
}

\maketitle

%\tableofcontents
\section{Introduction}
With ever-growing datasets both in the industry and in academia, ranking becomes
both more important and more interesting. In the 21th century gathering large
data sets is no longer considered novel. Most of our entertainment media is
online, and user ratings becomes the guiding star for quality. Ranking provides
what might be the simplest type of recommendation system: recommend the items
that score best on a global total ranking, independently of the recipient of the
recommendation.

\section{Ordering movie ratings in a dynamic setting}
\label{sec:intro_real}
To motivate our approach, we present our project through the following concrete
problem: An \texttt{IMDB} like recommendation service maintains a list of
movies, and lets users rank movies on some scale (say $\left[1,10\right]\cap
\mathbb{N}$). We want to provide the set of movies, sorted by user-supplied
ratings. It is likely that ratings of individual movies will span a large
portion of the valid range. I.e. movies might have a few maximum ratings, but
a low overall score, or vice versa. It is therefore necessary to use an
aggregate for the ranking, such as the average user-rating for each movie. For
simplicity, we assume that ratings cannot be changed or deleted, only added.
This makes it possible to maintain running averages, as opposed to the full set
of ratings, without integrity loss.

We assume that ratings are added frequently, and that queries are frequent. We
therefore consider the input --- the set of $(user, rating, movie)$ triplets ---
to be \textit{dynamic}. The input is dynamic in the sense that the \textit{true}
global ranking may change over time, with every added user-supplied rating. In a
static setting, where the global rank does not evolve over time, there are
simple algorithms that solve our problem. E.g. sorting the full data set would
do.
In a dynamic setting, where the data evolves over time,
those algorithms would need to be re-run to maintain current solutions (as the
underlying input changes, so does the global ranking, and hence the correct
solution). Hence, there may be better algorithms that maintain and evolve the
solution according to the input.

In our dynamic setting, the input is defined as a sequence (or multiset) $S =
\alpha_1, \alpha_2,\ldots,\alpha_{|S|+1}$. The universe $U$ is the set of movies
in our database, and $R$ is the set of valid ratings $R =
\left(\left[1,10\right]\cap \mathbb{N}\right)$. That is, $S$ consists of $|S|$
pairs $(j,r)\in U \times R$.

In this project, we explore different approaches to keeping a ranked data set
while receiving live updates to the data. We will start by analyzing the most
direct approach; ordering the full movie set on each query, or maintaining an
always-ordered data structure with the movies, updating it with each rating in
the input.

We explore different streaming algorithms, in an attempt to achieve sublinear
memory bounds.

We then present experimental results, measuring the error in the
global rank resulting from estimating rating averages. Finally, we go out on a
limb, describe a simple TerraSort-based \texttt{MapReduce} solution.

We choose algorithms to analyze based on time- and space- complexity, currency,
and the quality/correctness of the solutions in case of approximation
algorithms.
In particular it should be possible to achieve a memory bound lower than $|U|$ by
sacrificing correctness, achieve both by sacrificing currency, i.e. allowing the
output to become somewhat outdated as the underlying data set evolves. We note
that in our \texttt{IMDB} like scenario, the running time of getting a total
ordering is irrelevant. The running time per rating and per query is all that
matters, though they will often be closely related. 

%\begin{description}
%	\item[Parallelization] The use of parallelized sorting algorithms, such
%		as parallelized merge- or radix- sort. Parallization in algorithms is very
%		natural for sorting/selection. But it is still very overlooked. It might be
%		a challening to see the real impact of this - because we are affecting the
%		overall time consmuption by using it. Moreover usally we are adding more
%		complexity and overhead to our solution - by creating and mantaining parallel
%		task and jobs. This mean on of our point of emphasis will analysis of the
%		pros and cons carried by this approach.
%	\item[Approximation] We would like to explore the possibility of
%		performing an approximate sort. User--provided rankings are
%		often inconsistent. For instance, if a user ranks a move $m$
%		lower than a movie $m'$, does that mean he liked $m$ less, or
%		that he likes that genre less? It is unclear weather or not the
%		scale is linear, and the same user may have different
%		experiences on different days. This means that an \textit{exact}
%		ordering might not be necessary. If it provides a speedup, it
%		may be well worth it to perform a partial sorting of the
%		rankings --- such that a top--ten might really be a top--eight,
%		plus 12 and 14.
%	\item[Online sorting] Sorting is an inherently offline problem --- you
%		can not sort a set without having all the values. However, for
%		problems where the full data set is not immediately available,
%		onlineness can be achieved, or approximated, by sorting partial
%		data sets, and then in the end sorting the whole. Exploiting the
%		efficiency of certain sorting algorithms when dealing with
%		partially ordered data, to lessen the time spend waiting for the
%		data.
%
%\end{description}

\section*{General approach}
One of the basic and also efficient concepts is using a data structure which
will provide mechanism for extracting interesting data efficiently. 

\noindent \textbf{Binary Search Tree}

Creation of BST is an insertion of each element in $O(log{}n)$ time. As a result 
we get sorted structure of our inputs - accessing all of them takes $O(n)$.
Using this approach requires to store all input elements - which consumes $O(n)$ space. 
And if we want to find Low k elements of set we still need to travers all set.

\noindent \textbf{Heap and Min-Max Heap}

Ordering items using Heap takes $O(n\log{}n)$. Where creation of a heap takes
$O(n)$. Then we are traversing through heap n times using fuction which takes
$O(log{}n)$. Which results in $O(n + n\log{}n)$, which leads to our prevoiusly
stated complexity.

In scenario when we are only interested in Top and Low k elements of a set we might takes
different heap based approach called Min-Max Heap.

Provides properties of regular heap like a: 
\begin{itemize}
\item $O(n)$ built time,
\item $O(log{}n)$ for insertion and deletion
\end{itemize}

But even more important are properties carried by Min-Max variation. Which are 
guaranteeing us a constant time for Min and Max operations. The Top/Low(k) 
finding operation can be performed in $O(k\log{}k)$ time.

Using this approach requires to store all input elements. It takes $O(n)$ for 
large sets.

Based on : \texttt{M. D. Atkinson, J.R. Sack, N. Santoro, and T. Strothotte, 
Communications of the ACM, 
October, 1986, 
Min-Max Heaps and Generalized Priority Queues}

\subsection{Parallelization}
Utilizing power of multicore architecture in this scope is very tempting.
But our current approach is not supportive. By using data structure like Binary Tree
or Heap in a single thread scenario we are receiving multiple benefits like logarithmic time 
for insertion for each element or limited space usage. For each new element we have to only perform
insertion with logarithmic cost. For scenario when we are interested in intermidiate results it is 
a very big adventage against sorting algorithms like QuickSort or MergeSort.

But these algorithms have adventage over data structure based sorting - they are much
more prone to Parallelization. Depending on the architecture or use case even in light 
of need processing partial input it might be profitiable to resort our data multiple
time using benefits of multicore, multithreaded computing if time is our main constraint.





\section{A Stream Based Approach}
Streaming algorithms provide excellent solutions to many problems where data
sets are large enough that we wish (or need) to sacrifice exactness for low
memory usage and time consumption.
From \texttt{Ikonomovska\--Zelke}:
\begin{quote}
"Streaming algorithms drop the demand of random access to the input. Rather, the input
is assumed to arrive in arbitrary order as an input stream. Moreover, streaming algorithms
are designed to settle for a working memory that is much smaller than the size
of the input."
\end{quote}
To be precise, they require that the size of the working memory is sublinear in both the cardinality of the stream and the universe.
Due to this nature of streaming algorithms they are not commonly used for
problems that require analysing parts of non-constant, non-parameterized size
of the data set, for each given input. Hence an approach to
solve our problem --- sorting --- based on streaming algorithms will provide some interesting
trade-offs.

The definition of our stream, $S \in (U\times R)^{|S|}$ is the same as the
sequence described in the introduction. With this definition we get a strict
turnstile stream --- in fact the delta $r$ for \textit{every} stream element is
positive.

The simplest algorithm to solve our problem is then simply calculating the
normalized frequency vector for the stream, and sort it when queried.
However, this is not very satisfactory. The working memory is
sublinear in $|S|$, but linear in the universe size $|U|$. It
does not provide a current solution either, as we have to sort the frequency
vector when queried. On the positive side, the solution provided by the
algorithm is exact. To be precise, this algorithm would require $O(m)$ working
memory, and constant time for each stream item, $m$ being the number of distinct movies
in the stream, which we assume to be $|U|$. A query would then
require $O(m \log m)$ time.

The rest of this section discusses techniques that alleviate these problems
with different trade-offs.

\subsection{Order Maintenance}
In the simple algorithm, results were not \textit{current}, because every query
required a sort of the frequency vector --- which is long. If we allow ourselves
to use more than constant processing time per stream element, this problem can
be solved by maintaining an \textit{always sorted} data structure with pointers
into the frequency vector, such as a search tree.
This algorithm is equivalent to maintaining an ordered set of running averages,
and is thus the same as the online-sorting approach described previously.

Knowing that for most stream items $(j,r)$, the movie represented by
$j$ will already be in the ordered set, it might be possible to achieve
insertion time linear in the number of inversions needed to reorder the set,
though it is not clear that this should improve the $\log n$ insertion time in
binary search trees.

In summary, we get $O(\log n)$ processing time for each stream element, but
queries can now be performed in $O(n)$. This a very natural change from the
simple algorithm, that really only moves the required work from the time of
querying, to the time of input.

\subsection{Approximation based on sampling}
In addition to the lack of currency, the simple algorithm requires a lot of
memory. Not surprisingly, streaming algorithms lets us buy a lower memory
requirement, at the cost of exactness.

There seem to be two obvious approaches;
normal reservoir sampling over the stream, or sampling over the
movies, deliberately making sure that all movies are represented in the sample.
The later obviously fails to improve memory consumption, and is only suggested
because taking a random sample over the stream seems dangerous, as it might well
discard movies from the stream, by not picking any of their ratings for the
sample. As it turns out, this is not a big problem.

Although a uniformly random sample of ratings is not an answer to our original
problem, it does have some nice properties: As stated in
\sloppy{\texttt{Ikonomovska\--Zelke}} (p. 243); all $|S| \choose k$ possible
samples, where $k$ is the sample size, are equally likely to be our result. It
follows directly from this that popular ratings for a movie are more likely to
occur than unpopular ones. However, the likelihood of a movie occurring in the
sample similarly correlates to how many ratings it has, not -- as we would want
-- how high it's average rating is. In other words, reservoir sampling gives us
a random set of ratings, with no guarantee that the sample contains good movies.

The common reservoir sampling algorithm described in
\texttt{Ikonomovska\--Zelke} remembers $k$ samples $K_0..K_{k}$. After the first $k$, each
stream element \raggedright{$\alpha_i, k < i \leq |S|$}, replaces one of the
$k$ samples with probability $k/i$, choosing the sample to be replaced at
random.

The sampling approach solves our memory issues by parametrizing the memory
consumption. The algorithm uses $O(k)$ memory, independent of both $|U|$ and
$|S|$.

We can modify the reservoir sampling algorithm to keep running averages instead
of samples, modifying samples when observing a stream element that refers to a
movie already being monitored, and only replacing a sample when observing an
element that is not already being monitored (i.e. not currently in the set of
remembered movie samples). We then no longer get sampled ratings, but estimates
of the movie averages --- which is what we wanted. Maintaining running averages
can be thought of as keeping a frequency vector, and changing the stream so that
each element $(j, r)$ becomes $(j, r')$, where $r'$ is the change $r$ imposes on
the kept average for movie $j$. Despite the possibility of $r'$ being negative,
we are still in the strict turnstile model, as the average will never be
negative, regardless of what subset if $S$ we look at. Direct application of
this analogue is infeasible, as it would require $O(|U|)$ memory to keep
track off the counters necessary to calculate $r'$ from $r$. However, it would
be interesting to apply it with estimations of those counters.

The problem of missing movies persists however. If wish achieve the $O(k)$
memory bound, we can not hope to find the exact solution using sampling in this
way. However, if we limit our problem to find the top-l movies, we can.

%As described in \texttt{Ikonomovska\--Zelke}, we can adjust the quality of
%the result by adjusting $k$ as following.
%If 
\begin{quote}
	DRAFT NOTE: This would be a good place to analyze the quality of the
	solution --- in terms of $k$
	%In that case, it should be shown that it still holds (or not) after
	%changing the sampling method in the following paragraph
\end{quote}

If we alter our running-average sampling to replace the \textit{smallest}
sample, instead picking one at random, the probability of our $k$ samples
containing the top $l<k$ movies increases. \texttt{Metwally et al} present an
algorithm based on that idea. The algorithm mixes techniques from sampling and
estimation, to provide both top-k \textit{and} frequent-elements queries, albeit
for frequencies rather than averages.

The \textit{space-saving} algorithm presented in \texttt{Metwally et al} works
as follows:
To support the eviction of the smallest sample, as well as the queries on the
algorithm, the \textit{space-saving} algorithm keeps the samples $(U\times
\mathbb{N})^k$ in non-increasing order of frequency. We let $K_i$ denote the
frequency of the $i^{th}$ sample in this order, hence $K_k = \min_j\pi_2(K_j)$.
When a monitored movie is observed, it's counter is incremented. When a
non-monitored movie $j'$ is observed, the movie replaces the $k^{th}$ sample
$(j,K_k))$. Since $j'$ may at at this time have
been observed at most $K_k+1$ times, the new sample is added as $(j',K_k+1)$. This
introduces an element of estimation, trying to make up for increments lost by
previous evictions from the sample-set. For each sample, the maximum
overestimation $\varepsilon_i$ is tracked. $\varepsilon_i = K_k-1$ after the
sample is replaced (equal to $K_k$ before the sample is replaced). The algorithm
cleverly introduces error when the replacement occurs, and is hence more likely
to err on infrequent elements than on frequent ones. I.e. the ones we are least
interested in.

\texttt{Metwally et al} Proves several theorems that are important for our use
case: \textit{(Adapted to our problem definition. Numbering corresponds to that
of the paper)}
\begin{description}
\item[Metwally Lemma 1] \hfill \\
	The length $|S|$ of the stream is equal to the sum of the sample
	frequencies.
	$|S| = \sum_{i\leq k}K_i$

\item[Metwally Lemma 2] \hfill \\
	$K_k \leq \lfloor \frac{|S|}{k} \rfloor$ 

\item[Metwally Lemma 3] \hfill \\
	For any sample: $0 \leq \varepsilon_i \leq K_k$ i.e $f_i \leq f_i
	+ \varepsilon_i = K_i \leq f_i+K_k$ where $f_i$ is the actual frequency
	of the movie we estimate to have rank $i$.

\item[Metwally Theorem 1] \hfill \\
	Let $F_i$ denote the actual frequency of the movie with actual rank $i$.

	Any movie with $F_i > K_k$ must exist in the sample-set. I.e any movie
	with an actual frequency higher than the lowest estimated frequency in
	the sample, must be in the sample.

\item[Metwally Theorem 2] \hfill \\
	Whether or not the movie with actual rank $i$ occupies the $i^{th}$
	position in $K$, $K_i \geq F_i$

\end{description}

As mentioned, the \textit{space-saving} algorithm is intended for
estimating frequencies, and needs modification to work with averages. A first
thought might be to run two instances in parallel, estimating the count of
ratings and sum of ratings respectively (equivalent to frequencies in the cash
register and turnstile model respectively). This clearly does not work, as the
set of most frequently rated movies is not necessarily similar to the set of
movies with a high sum of ratings. Instead, we can adapt the algorithm to
maintain running averages instead of frequencies. This presents a problem
however. The proofs for the presented lemmas and theorems depend on the fact
fact that each replacement in the sample-set increments the counter by $1$. But
incrementing by $1$ will not provide us with running averages. Hence, keeping
running averages will break the proofs. 

\begin{quote}
	DRAFT NOTE: Now actually adapt it
	to the running averages, and analyse it.
	Also mention it's relation to sketching.
\end{quote}

\subsection{Approximation based on Sketching}

\begin{quote}
	DRAFT NOTE: Consider moving this to before the sampling section.
	And maybe extract the general stuff to the (or a new) super section.
\end{quote}
Sketching lets us sacrifice exactness for lower memory consumption, without
having to worry about loosing entire movies from the solution. We pay for this
with an error bound that depends on $|S|$.

The count-min sketch algorithm as described in \texttt{Ikonomovska\--Zelke}
estimates the frequency vector of the stream, using
$O(\log(1/\delta)/\varepsilon + \log n \cdot \log(1/\delta))$ working memory.
Where $\varepsilon$ determines the expected error $\varepsilon \cdot |S| /2$, of
each frequency and $\delta$ is the probability of the actual error exceeding
that bound. The error is guaranteed to be positive i.e. the algorithm cannot
underestimate the actual frequencies. A simple adaption of the algorithm to our
problem (estimating averages, not frequencies) is to run two instances of the
algorithm in parallel, estimating respectively the sum and frequency of ratings
for each movie. We denote the actual sum and frequency for a movie $j$ by $R_j$
and $F_j$, and the estimates produced by the algorithm $\bar{R_j}$ and
$\bar{F_j}$ We can then get an estimate of the average rating for a movie $j$ by
$\bar{R_j}/\bar{C_j}$.
If we use the same set of hashing functions and
vector lengths for sketching $R_j$ and $C_j$, we see that
$R_j-\bar{R_j} \le F_j-\bar{F_j}$. Hence, the resulting error is
positive, unless ... . This guarantee only holds because our ratings are
defined to be positive. This also shows
$$\frac{R_j}{F_j} \le \frac{\bar{R_j}}{\bar{F_j}} \le \frac{\epsilon_r \cdot
R_j}{\epsilon_f \cdot F_j}$$
Where $\epsilon_f$ is our error bound $\varepsilon \cdot |S|/2$, and
$\epsilon_r$ is $\epsilon_f$ times the maximum rating (10, in our case).

\begin{quote}
	DRAFT NOTE: this is a very simplified analysis.
	It would be good to find the actual expectancy of this. Seems like it
	should be easy, but apparently I'm retarded.
\end{quote}


\section{Experiment}
While the modified Count-min sketch algorithm provides an error bound for the
estimated movie averages, what we are really interested in is how much it
affects the ordering of the movies when sorted by these estimated averages
compared to the real averages. In this section, we evaluate the modified
Count-min sketch algorithm by running it on real-life data, and comparing the
order it produces with the correct order. 

We quantify the error in ordering by the Kendall tau distance. For a permutation
of movies $\pi$, we say that $x <_\pi y$ if $x$ comes before $y$ in $\pi$. The
Kendall tau distance between two orderings of our movie set is then $\mathrm{KT}
\left(\pi_1,\pi_2\right) = \left|\{x,y\}: x <_{\pi_1} y \wedge y <_{\pi_2}
x\right|$. The Kendall distance is often normalized to the maximum number of
inversions between the orderings, $n(n-1)/2$ for $n$ elements. We expect random
orderings to have a normalized Kendall tau distance of $\frac{1}{2}$, so
only results with a lower distance can be considered useful.

\subsection{Results}
Our data is sampled from the \textit{Netflix
Prize}(http://academictorrents.com/details/9b13183dc4d60676b773c9e2cd6de5e5542cee9a) data set.
The full data set contains more than 100 million user ratings, for more than 17
thousand movies.
We test the algorithm on three samples of the \texttt{Netflix} data:
\begin{itemize}
	\item \textit{100m ratings} contains all ratings for the $13726$ most
		frequently rated movies. Roughly 100 million ratings total,
	\item \textit{50Mmin ratings} contains all ratings for the $17146$ least
		frequently rated movies. Roughly 50 million ratings total,
	\item \textit{50m ratings} contains all ratings for the $611$ most
		frequently rated movies. Roughly 50 million ratings total,
	\item \textit{min10K ratings} contains all ratings for the $2042$ movies that have more
		than $10.000$ ratings.

\end{itemize}

\pgfplotsset{scaled x ticks=false}
\begin{center}
\begin{tikzpicture}
\begin{axis}[
	title=Count-min order error,
	xlabel={Error bound $\varepsilon$},
	ylabel={KT. norm.},
	legend pos=south east,
	xticklabel style={
		/pgf/number format/.cd,
		fixed,
		fixed zerofill,
		precision=2,
		/tikz/.cd
	},
]
\addplot table [y=100M,x=E]{allresults};
\addlegendentry{\textit{100M ratings}}
\addplot table [y=50M,x=E]{allresults};
\addlegendentry{\textit{50M ratings}}
\addplot table [y=min10K,x=E]{allresults};
\addlegendentry{\textit{min10K ratings}}
\addplot table [y=50Mmin,x=E]{allresults};
\addlegendentry{\textit{50Mmin ratings}}
\end{axis}
\end{tikzpicture}
\end{center}

\subsection{100m ratings}		
As we can see the first plot - $100M raitings$ show how badly algorithm worked for this data set.
The Kendall-Tau value starts from $0.35$ and rapidly grows in direction to $0.5$. Which is very bad.
This behaviour is caused by length of the stream - longer it is more worse results we get. And by the fact
that the 100 millions is basicaly our entire data set. So the number of raitings per movie differs.

\subsection{50Mmin ratings}	
Same trends as in $100m raitings$ is present in $50Mmin raitings$. Again we are working on relatively big set. But this time it contains
only the movies which were least rated. Even with the fact we are working on half smaller set we still gets poor results. We cannot 
achieve good result with set which contains rarely rated movies.

\subsection{50m ratings}
Again we are working on smaller set. But this time we have the most frequently rated movies. Difference in shape of plot is huge.
It starts from Kendall-Tau distance equal $0$ where the previous two data sets started with 0.3 more. Even for higher error bounds 
the results ale still better. 

\subsection{min10K ratings}
Last which produced best results. Which contains only $10$ tousands movies with high amount of raitings. Shape of plot is similar to 
$50min raitings$ but slight better by average $0.20$ normalized Kendall-Tau distance. Probably thanks to fact the stream is smaller than
$50m raitings$. 



\subsection{Implementation}
The implementation can be found in \texttt{code} folder that came with this
report. We Note that the implementation does not live up to the performance
bounds stated in section \ref{sec:sketching}. This is not a problem, since we
are assessing the correctness of the algorithm, not it's throughput. The
implementation differs in two ways: To emulate querying all data points, a list
of all observed movies is kept, requiring $O(|U|)$ extra memory. Furthermore the
ratings are sorted when queried, rather than maintaining the ordering
dynamically. Again, this does not effect the experiment.


\section{Map Reduce Approach}
In light of constantly growing datasets solving even 
most basic computational problems becomes more difficult and complex.
One of the attempts to adress this need is a model called MapReduce.
Which allows processing in a parallel, distributed manner.

\section{Two subproblems}
Our problem of creating ranking of movies based on the user raiting 
consist two subproblems. First one is creation of averages movie rate.
Second is a putting them in order. 


\subsection{Sorting average raitings}
In 2008 Owen O'Malley won TeraByte sorting competion using Apache Hadoop
- Java framework which process data using MapReduce. TeraSort offers
high scalability whitout adding overhead and overall complexity.

In paper Minimal MapReduce Algorithms TeraSort was classified as 
minimal MapReduce algorithm. Which definition and properties are given
here:

From \texttt{Minimal MapReduce Algorithms}:
\begin{quote}
Denote by $S$ the set of input objects 
for the underlying problem. Let n, the problem cardinality,
be the number of objects in $S$, and $t$ be the number of machines
used in the system. Define $m = n/t$, namely, $m$ is the number
of objects per machine when $S$ is evenly distributed across the
machines. Consider an algorithm for solving a problem on $S$.
We say that the algorithm is minimal if it has all of the following
properties.

\begin{itemize}
\item Minimum footprint: at all times, each machine uses only
O(m) space of storage.
\item \emph{Bounded net-traffic}: in each round, every machine sends
and receives at most $O(m)$ words of information over the
network.
\item  \emph{Constant round}: the algorithm must terminate after a
constant number of rounds.
\item \emph{optimal computation}: every machine performs only
$O(T_seq /t)$ amount of computation in total (i.e., summing
over all rounds), where $T_seq$ is the time needed to solve the
same problem on a single sequential machine. Namely, the
algorithm should achieve a speedup of t by using t machines
in parallel. 
\end{itemize}
\end{quote}


Which perfectly denotes main advantages of the TeraSort. The minimum footprint property guarantees
that each of our machines will use the same amount of memory. Which makes profiling
and planning infractures much easier. The same hold for network throughput - by knowing
upper bound how much data will be transferred in the process we easily can track bottlenecks
and scale our architecture.
By having our load balanced it also benefits to performance. We can easily adopt amount of machines
taking part in the computation which allows to easily maintain balance between speed and costs.
Of course we could just use Quicksort instead of TeraSort - but the constant amount of rounds property
is highly in favour of TeraSort. It guarantees better effiency and predictiable time of execution.


\subsection{Calculating average ranking}
To calculate Moving Average of each movie we again will used TeraSort. 
But with modified Round 2. Where instead of sorting items in Reduce phase
we will calculate average. Since TeraSort by design is prepared to execute 
sorting procedure in this particiular phase transition is very simple.

Thanks to fact that each machine contains part of data. And this particiular part contains movies from
a given range it simplies process a lot. And makes it faster because we do not need to iterate over entire 
input just in specific part.
Our Moving Average calculation procedure is very simple. Similar to General Approach
we are using HashMap. Where every new encountred movie is added to HashMap. If it is already 
present we are adding next raiting to property which holds sum of it and also incrementing counter.
Which are used to calculation average.









\end{document}
