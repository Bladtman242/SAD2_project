\documentclass[a4paper, titlepage]{report}

\usepackage[utf8]{inputenc}
\usepackage{courier} % Required for the courier font
\usepackage[bookmarks]{hyperref}

%redefine percentage sign to be a little smaller
\let\oldpct\%
\renewcommand{\%}{\scalebox{.9}{\oldpct}}
\begin{document}

\title{Problem Description}
\author{
	Radoslaw Niemczyk
	\\\texttt{radn@itu.dk}
	\and
	Sigurt Bladt Dinesen
	\\\texttt{sidi@itu.dk}
}

\maketitle

%\tableofcontents

\section*{Ranking a large data set}
Ranking has become an increasingly important problem, with ever-growing datasets
both in the industry and in academia, the ability to work with large ranked
sets, and to do so fast.

Ranking is used commercially to provide customers with proposed subsets of
products, to make it easier to choose a product of interest. For this it might
be valuable to have the entire set of products ordered by rank, but it may be
sufficient to have a \textit{top-n} list, of the $n$ highest (and/or lowest)
ranked items. 

This project is concerned with sorting a large set of user rankings. Following
is a list of techniques that we deem promising, and would like to explore the
effectiveness of:

\begin{description}
	\item[Parallelization] The use of parallelized sorting algorithms, such
		as parallelized merge- or radix- sort.

	\item[Approximation] We would like to explore the possibility of
		performing an approximate sort. User--provided rankings are
		often inconsistent. For instance, if a user ranks a move $m$
		lower than a movie $m'$, does that mean he liked $m$ less, or
		that he likes that genre less? It is unclear weather or not the
		scale is linear, and the same user may have different
		experiences on different days. This means that an \textit{exact}
		ordering might not be necessary. If it provides a speedup, it
		may be well worth it to perform a partial sorting of the
		rankings --- such that a top--ten might really be a top--eight,
		plus 12 and 14.

	\item[Online sorting] Sorting is an inherently offline problem --- you
		can not sort a set without having all the values. However, for
		problems where the full data set is not immediately available,
		onlineness can be achieved, or approximated, by sorting partial
		data sets, and then in the end sorting the whole. Exploiting the
		efficiency of certain sorting algorithms when dealing with
		partially ordered data, to lessen the time spend waiting for the
		data. 

\end{description}

\end{document}




